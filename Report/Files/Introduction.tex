\subsection{The Elk River problem}
The Elk River is a situated in central West Virginia, USA. Elk River watershed is a relatively big but low populated area with Charleston being the only big city. It is characterized by vast natural resources and less developed industry and infrastructure. In 2014, one of the biggest employers in the area was involved in a major ecological incident, namely a chemical spill, which led to closure of the plant, leaving many people jobless and making impact on the local flora and fauna. Since then, the area has not been able to resume economic development, but still it attracted a lot of tourists, especially fishers who like to enjoy many fishing spots that the river offers. After analyzing the current situation in the area and all groups of people living and visiting the area, it becomes evident that one of the things that is most probable to succeed in improving the well-being of all groups could be improving the tourist offering. The goal of this document is to further explain the platform using user-centered approach.
\subsection{Stakeholders}
Stakeholders are the locals, whose economy is struggling since the 2014 spills. Many of them will be able to establish a job position with a stable income thanks to our service, with the possibility of working part time as a second job. Our platform will also boost the economy of the area by attracting tourists, so many activities already in place will indirectly benefit from the service, and others will born.
\subsection{Definitions and acronyms}
What follows is the list of all the main definitions and acronyms used in the document.
\subsubsection{Definitions}
\begin{itemize}
\item \textbf{Reservation}: data referring to the wish of the user to have the specified service at that time for himself.
\item \textbf{System}: All the software needed to deliver every functionality needed.
\end{itemize}
\subsubsection{Acronyms}
\begin{itemize}
\item \textbf{BPMN}: Business Process Model and Notation
\item \textbf{SDK}: Software Development Kit
\item \textbf{API}: Application Programming Interface
\item \textbf{DB}: Database
\item \textbf{DBMS}: Database Management System
\item \textbf{UID}: Unique Identifier
\item \textbf{URL}: Uniform Resource Locator
\item \textbf{UI}: User Interface
\end{itemize}