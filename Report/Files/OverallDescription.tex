\externaldocument{Files/PrototypeMobileApplication}
\externaldocument{../DIL_Report}
\subsection{Elk River area resources}
The Elk River area doesn't have particular infrastructures outside of asphalted roads connecting one small countryside village to the others, more or less coasting the entire length of the river. Therefore the main and basically only resources are the ones given by the environment itself, meaning that the rural composition of the area allows for sports and activities that aren't possible in a city center, such as fishing, rafting, camping, hunting etc...\\\\
Some structures like restaurants and also hotels/farmhouses where one could spend a weekend are already present in the area, meaning that they would work as an incentive for people living further away to visit the area anyway without the need of doing two long drives in one single day given the lack of public transport outside of the main cities.
\subsection{Proposed project solution}
As we will discuss more in detail during the next sections, we decided to opt for a solution involving fishing activities, given that some of the few records of what people do while visiting the Elk River area is indeed fishing related.\\
It should be noted that the system developed could be easily scaled up to accommodate other activities but we chose to keep it simpler given that it's meant to be a first prototype and it seemed more logical to focus on what looked like the main venture.
\subsection{Project structure}
\subsubsection{Data Analysis}
We conducted some data analysis in \autoref{sect:Data Analysis} to find what the potential customers may need and/or want, the main techniques used were:
\begin{enumerate}
\item Clustering
\item Classification
\end{enumerate}
\subsubsection{Prototype}
The final system is going to be divided in two main components:
\begin{enumerate}
\item Mobile application for phones and tablets.
\item Backend structure to support the functioning of the service.
\end{enumerate}
While the backend structure is needed for the functioning of the service provided, the user will never interact with it but will ever only see and use the mobile application. A more detailed view will be explained in \ref{ApplicationStructure}.
\clearpage
\subsection{Product Functions}
\begin{enumerate}
\item Register to the system with email and password or other services (i.e. Gmail).
\item Logging into the service.
\item Manage the information of an account.
\item Create and delete a reservation.
\item Provide a communication system between customers and employees.
\end{enumerate}
\subsection{User Characteristics}
The users interested in using the system should be at least familiar with the concept of using a smartphone in the day to day routine without needing any technical competence regarding the topic, they must be aware of the laws regarding fishing, know how to traverse the local environment and have basic first aid knowledge if they wish to work as fishing experts.
\subsection{Goals}
The project is designed to satisfy the user needs, or (in other words) to achieve certain specific \emph{Goals} stated in the following list.
\begin{itemize}
\item Allow anyone that owns a smartphone to become a registered user of the service.
\item Allow locals to find a new source of income.
\item Allow tourists to have direct access to experts of the area.
\item Allow tourists to find fishing spots suggested by experts.
\item Bring outside people to the Elk River area with the intent of boosting the economy through fishing and tourism.
\end{itemize}
