\externaldocument{Files/PrototypeMobileApplication}

\subsection{Proposed project solution}
\subsubsection*{Elk River area resources}

\subsubsection*{Project structure}
The final system is going to be divided in two main components:
\begin{enumerate}
\item Mobile application version for phones and tablets.
\item Backend structure to support the functioning of the service.
\end{enumerate}
While the backend structure is needed for the functioning of the service provided, the user will never interact with it but will ever only see and use the mobile application. A more detailed view will be explained in \ref{ApplicationStructure} \\

\subsection{Product Functions}
\textbf{TO BE EXPANDED}
\begin{enumerate}
\item Register to the system with email and password.
\item Logging to the service.
\item Manage the information of an account.
\item Create or delete a reservation.
\end{enumerate}
\par
\subsection{User Characteristics}
The users interested in using the system should be at least familiar with the concept of using a smartphone in the day to day routine without needing any technical competence regarding the topic, they must be aware of the laws regarding fishing, they should also know how to traverse the environment and have basic first aid knowledge if they wish to work as fishing experts.
\newpage
\subsection{Goals}
The project is designed to satisfy the user needs, or (in other words) to achieve certain specific \emph{Goals} stated in the following list.\\
\textbf{TO BE EXPANDED}
\begin{enumerate}[label=Goal.\arabic*:]
\item Allow anyone that owns a smartphone to become a registered user of the service.
\item Create and later manage reservations.
\end{enumerate}
